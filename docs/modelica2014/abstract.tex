%sample file for Modelica 2011 Abstract page

\documentclass[11pt,a4paper]{article}
\usepackage{graphicx}
\graphicspath{{fig/}}
% uncomment according to your operating system:
% ------------------------------------------------
%\usepackage[latin1]{inputenc}    %% european characters can be used (Windows, old Linux)
\usepackage[utf8]{inputenc}     %% european characters can be used (Linux)
%\usepackage[applemac]{inputenc} %% european characters can be used (Mac OS)
% ------------------------------------------------
\usepackage[T1]{fontenc}   %% get hyphenation and accented letters right
\usepackage{mathptmx}      %% use fitting times fonts also in formulas
\usepackage{lmodern,amsmath,mathptmx,url}      %% recommended for readable pdf
% do not change these lines:
\pagestyle{empty}                %% no page numbers!
\usepackage[left=35mm, right=35mm, top=15mm, bottom=20mm, noheadfoot]{geometry}
%% please don't change geometry settings!
\setlength{\parindent}{10pt}
\setlength{\parskip}{3 mm}

% some additional packages
\usepackage{color}
\usepackage{indentfirst}
\usepackage[hidelinks=true]{hyperref}
\usepackage[hidelinks=true]{hyperref}
\hypersetup{%
  pdftitle = {impact -- A Modelica Package Manager},
  pdfauthor = {Michael Tiller and Dietmar Winkler},
  pdfsubject = {10th International Modelica Conference 2014},
  pdfkeywords = {Modelica, package manager, GitHub, dependency management, Python}}
\usepackage[backend=bibtex,sorting=none]{biblatex}
\addbibresource{impact}

% usefull commands
\newcommand{\myr}{\textsuperscript{\textregistered}}
\newcommand{\ud}{\mathrm{d}}
\newcommand{\matx}[1]{\mathbf{#1}}
\newcommand{\impact}{\texttt{impact}} % impact is going to get used quite a lot :)
\newcommand{\code}[1]{\texttt{#1}} % make quoting code text a bit simpler

% begin the document
\begin{document}
\thispagestyle{empty}

\title{\impact\ -- A Modelica\myr\ Package Manager}

\author{Michael Tiller\\
  \href{http://xogeny.com}{Xogeny Inc.}, USA\\
  \href{mailto:michael.tiller@xogeny.com}{\nolinkurl{michael.tiller@xogeny.com}} %
  \and Dietmar Winkler\\
  \href{http://www.hit.no}{Telemark University College}, Norway\\
  \href{mailto:dietmar.winkler@hit.no}{\nolinkurl{dietmar.winkler@hit.no}}}
\date{} % <--- leave date empty
\maketitle\thispagestyle{empty} %% <-- you need this for the first page

To manage complexity, modern programming languages use organizational
units to group code related by some common purpose.  Depending on the
programming language, these units might be called libraries, packages
or modules.  But they all attempt to encapsulate functionality to
promote modular code and reusability.  For the remainder of this
paper, we will simply refer to these organizational units as
\emph{packages} (as they are called in Modelica).

Also common to many modern programming languages are tools to manage
these packages.  These tools are generally called \emph{package
  managers} and they allow developers to quickly ``fetch'' any
packages they may need for a given project.  The main functions of
package managers are to allow developers to search, install, update
and uninstall packages with a simple command-line or graphical
interface.  In the Java world, the most common package manager is
\code{maven}.  For Python, tools like
\code{easy\_install}\cite{easy_install} and \code{pip}\cite{pip} are
used for managing packages.  For client-side web development,
\code{bower} is used.  For server-side JavaScript, the tool of choice
is \code{npm}\cite{npm}.  For compiled languages, these package
managers often include some additional build functionality as well.

This paper introduces \emph{\impact}, a package manager for
Modelica. Using \impact, Modelica users and developers can quickly
search for, install and update Modelica libraries.  In this paper, we
will discuss the functionality provided by \impact.  In addition, we
will discuss how the functionality was implemented.  As part of this
we will discuss the importance of collaborative platforms, like
\href{https://github.com}{GitHub}\cite{github} in our case, for providing
a means for collecting, curating and distributing packages within a community
of developers.

The \impact\ package manager is provided to the Modelica community as
a free, open-source tool.  Furthermore, the protocols involved are all
documented and we encourage tool vendors to integrate them into their
own tools so they can provide the same searching, updating and
installation capabilities that the command-line tool provides.

\printbibliography

\end{document}
