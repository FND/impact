% sample file for Modelica Conference paper

\documentclass[11pt,a4paper,twocolumn]{article}
\usepackage{graphicx}
\graphicspath{{fig/}}
\usepackage[T1]{fontenc}
\usepackage[british]{babel}      % some british specific settings
\usepackage[utf8]{inputenc}    %% european characters can be used
\usepackage{lmodern,amsmath,mathptmx,url}      %% recommended for readable pdf
\pagestyle{empty}                %% no page numbers!
\usepackage{geometry}            %% please don't change geometry settings!
\geometry{left=20mm, right=20mm, top=25.4mm, bottom=25mm, noheadfoot, columnsep=8mm}
\parindent0pt

\usepackage[backend=bibtex,sorting=none]{biblatex}
\addbibresource{impact}

% some additional packages
\usepackage{color}
\usepackage[hidelinks=true]{hyperref}

% usefull commands
\newcommand{\myr}{\textsuperscript{\textregistered}}
\newcommand{\ud}{\mathrm{d}}
\newcommand{\matx}[1]{\mathbf{#1}}
\newcommand{\impact}{\texttt{impact}} % impact is going to get used quite a lot :)
\newcommand{\code}[1]{\texttt{#1}} % make quoting code text a bit simpler

\begin{document}

\title{\textbf{{\small Modelica'2014}\\
    \impact -- A Modelica\myr\ Package Manager}}

\author{Michael Tiller\\
  \href{http://xogeny.com}{Xogeny Inc.}, USA\\
  \href{mailto:michael.tiller@xogeny.com}{\nolinkurl{michael.tiller@xogeny.com}} %
  \and Dietmar Winkler\\
  \href{http://www.hit.no}{Telemark University College}, Norway\\
  \href{mailto:dietmar.winkler@hit.no}{\nolinkurl{dietmar.winkler@hit.no}}}
\date{} % <--- leave date empty
\maketitle\thispagestyle{empty} %% <-- you need this for the first page

\section*{Abstract}

To manage complexity, modern programming languages use organizational
units to group code related by some common purpose.  Depending on the
programming language, these units might be called libraries, packages
or modules.  But they all attempt to encapsulate functionality to
promote modular code and reusability.  For the remainder of this
paper, we will simply refer to these organizational units as
\emph{packages} (as they are called in Modelica).

Also common to many modern programming languages are tools to manage
these packages.  These tools are generally called \emph{package
  managers} and they allow developers to quickly ``fetch'' any
packages they may need for a given project.  The main functions of
package managers are to allow developers to search, install, update
and uninstall packages with a simple command-line or graphical
interface.  In the Java world, the most common package manager is
\code{maven}.  For Python, tools like
\code{easy\_install}\cite{easy_install} and \code{pip}\cite{pip} are
used for managing packages.  For client-side web development,
\code{bower} is used.  For server-side JavaScript, the tool of choice
is \code{npm}\cite{npm}.  For compiled languages, these package
managers often include some additional build functionality as well.

This paper introduces \emph{\impact}, a package manager for
Modelica. Using \impact, Modelica users and developers can quickly
search for, install and update Modelica libraries.  In this paper, we
will discuss the functionality provided by \impact.  In addition, we
will discuss how the functionality was implemented.  As part of this
we will discuss the importance of collaborative platforms, like
\href{https://github.com}{GitHub}\cite{github} in our case, for providing
a means for collecting, curating and distributing packages within a community
of developers.

The \impact\ package manager is provided to the Modelica community as
a free, open-source tool.  Furthermore, the protocols involved are all
documented and we encourage tool vendors to integrate them into their
own tools so they can provide the same searching, updating and
installation capabilities that the command-line tool provides.

\paragraph{Keywords:}
\emph{Modelica, package manager, GitHub, dependency management, Python}

\section{Introduction}
\label{sec:intro}

It is increasingly the case that the adoption of new technologies
hinges on automating away the tedious tasks required to learn and
adopt these new technologies.  For programming languages or
frameworks, this means streamlining the process by which libraries can
be found and installed.

For nearly all modern languages, this issue of ``package management''
has reached the point where it is almost an element of language
design.  The Java world has the \code{maven} tool, Scala has
\code{sbt}, Node.js has \code{npm} and the Go language includes
built-in support for package management via its the command line
\code{go} compiler.

In the Modelica world, this issue has been largely overlooked.
Although there have been proposals for formats to list network
accessible libraries, these efforts have remained mere proposals
without any concrete functionality.  The \impact\ project was inspired
by the Bower\cite{Bower} project's approach.  This lightweight,
\code{git}-centric approach (discussed in
Section~\ref{sec:candidates}) turned out to be relatively
straightforward to implement and provides functionality otherwise
unavailable in the Modelica world.

The goal of this project is to provide the same basic package
management features found in most package managers.  These features
will be discussed in detail in subsequent sections of this paper (see
Section~\ref{sec:command_line}).  Our goal is to lower the barrier for
users to find, install and update libraries.  At the same time, we
expect that the \impact\ tool itself will be just as easy to install
as the libraries it supports.

The problem that \impact\ addresses is making installation of Modelica
packages as easy as possible.  There are actually three important
aspects to the problem.  The most obvious aspect is providing a
command-line interface that can be used by users to easily install not
just a given Modelica library, but \textbf{also its dependencies}.
However, such a command line tool must rely on the second aspect which
is the avialability of a centrally served, up to date index of
packages.  The final aspect is making it easy for library developers
\textbf{to publish} their libraries in such a way that they are
available to other Modelica users through the \impact\ package
manager.  Each of these aspects will be discussed as part of this
paper.

It is worth noting that while \impact\ can handle dependences, it
does not solve some of the problems currently inherent in Modelica.  At
the moment, dependencies between packages are described by individual
versions.  The result is that these dependencies can create brittle
chains which are not always possible to satisfy. The logic for
unconvering dependencies in \impact\ is very simple.  It merely
identifies any dependencies explicitly listed by the library and then
attempts to find that version of those packages within the
\impact\ package index. Hopefully the Modelica annotations to express
dependencies will be refined to support a richer set of relationships.
If so, the logic used by \impact\ to identify and install these
dependencies can be extended to support this improved expressiveness.

\section{Command Line Interface}
\label{sec:command_line}

\subsection{Installation of \impact}
\label{sec:install}

The \impact\ tool is available from ``PyPI''\cite{pypi} and can be
installed by running either

% Start your code-block
\begin{verbatim}
$ pip install impact
\end{verbatim}
or
\begin{verbatim}
$ easy_install impact
\end{verbatim}

As an alternative one can also download the sources from
\url{https://github.com/xogeny/impact} or
\url{https://pypi.python.org/pypi/impact},
unpack and run
\begin{verbatim}
$ python setup.py install
\end{verbatim}

in order to install it.

\subsection{Searching} % Make each command a subsection
\label{cmd:search}

Searching for librararies is done by executing:

\begin{verbatim}
$ impact search <search term>
\end{verbatim}

This will return a list of all package whose names and/or description
strings contain the ``\code{<search term>}''. The returned list also
contains the URL where the Modelica package is hosted.

The output can also be made more verbose with:
\begin{verbatim}
$ impact search -v <search term>
\end{verbatim}
which, in addition, will return the description string and the
available versions.

\subsection{Installing packages} % Make each command a subsection
\label{cmd:install-pkg}
Once a package of interest is found using \code{search}, it can be
installed by executing:
\begin{verbatim}
$ impact install <package name>
\end{verbatim}
This will then fetch not only the package itself and extract it in a
configurable target directory but \textbf{it will also fetch the
  dependencies of the packages} as long as those are available to
\impact.

If several versions of a package are available, \impact will choose
the latest one. If this is not desirable then one can also specify the
version explicitly. For example, in order to install Modelica version
2.2.2 one would execute the command:
\begin{verbatim}
$ impact install Modelica#2.2.2
\end{verbatim}

Just like for the \code{search} sub-command there is also a verbose
switch ``\code{-v}'' available for \code{install} which will give
information on what versions are available, which version is going to
be installed, what the dependencies are and where the version will be
downloaded from.

In addition there is also a ``\code{[--dry-run|-d]}'' option available
which does not download or extract any files but will simply report
what \impact\ \emph{would} do.  It makes sense to use the ``dry-run''
option always in combination with the ``verbose'' option.

% Discuss "refresh" command here?  I assumed no when I wrote the
% section on private repositories.  But now I'm struggling with how
% to introduce the topics in section sec:index if we haven't talked
% about how indices are made.

% I agree, we should not talk about this "plumbing" command here.

\section{Candidate Packages}
\label{sec:candidates}

The question that might arise now is, how does \impact\ know what
packages and which versions of those packages are available.

\subsection{Making packages visible}
\label{sec:collection}
The Modelica Association (MA) has always maintained a list of
available Modelica libraries on their website under
\url{https://modelica.org/libraries}.  Initially, the list was a
static web page which listed the different packages and their latest
version as submitted to the webmasters of the MA.  Keeping the list of
packages up-to-date was a manual job for both the website maintainers
and package developers.

In spring 2013, all the free packages listed on
\url{https://modelica.org/libraries} were moved to individual
repositories on GitHub. Third-party packages can be found under
\url{https://github.com/modelica-3rdparty} and packages by the MA
under \url{https://github.com/modelica} This had the following
benefits:
\begin{itemize}
\item Package developers can access their package repository directly without
having to involve the webmasters of the MA thus submitting updates any time.
\item All packages now have an individual issue tracker and version control
service in place.
\item The webmasters of the MA can now collect the latest information on all
the packages automatically in order to generate an up-to-date listing on
\url{https://modelica.org/libraries} (more on this later in
Section~\ref{sec:gh-api}).
\end{itemize}

\subsection{Semantic versioning}
\label{sec:semver}
One thing that is important when trying to build up a package manager
that can also handle version dependencies is the need for a proper
approach to version numbering.

We decided to base our package manager on a system called Semantic
Versioning\cite{semver}.  As a result, package developers are strongly
encouraged to use semantic versioning so that they are compatible with
\impact.  This has the additional benefit of being a well-documented
and logical approach.

Semantic versioning has the simple rule of:

{\small
\emph{Given a version number MAJOR.MINOR.PATCH, increment the:
  \begin{enumerate}
  \item MAJOR version when you make incompatible API changes,
  \item MINOR version when you add functionality in a backwards-compatible
    manner, and
  \item PATCH version when you make backwards-compatible bug fixes.
\end{enumerate}
}
}

In addition, pre- and post-releases are also taken care of in this
system, details can be found in the manual page\cite{semver}.

% Discuss the Github API, how we walk it, what we produce from it and
% where we store the results
% MIKE: I wonder if I should talk already here about the result files
% when we do mention much later in the sec:index. What do you think?

\subsection{GitHub API}
\label{sec:gh-api}
Having all packages available as GitHub repositories means that we can
use the \emph{GitHub API v3}\cite{gh-api} in order to collect data
about those packages.  All API access is over HTTPS, and accessed from
the \url{api.github.com} domain.  All data is sent and received as
JSON\cite{json}.

For example if one visits:
\url{https://api.github.com/users/modelica/repos } a verbose list
containing a series of information of all repositories that exist
under the user \code{modelica} is returned in JSON format.  This
includes also a new API-url for retrieving the tags of a specific
repository.  For example, by visiting
\url{https://api.github.com/repos/modelica/Modelica/tags} we get a
list of all tags for the Modelica Standard Library repository
including download links for a zipped version of the tagged source
code.

As mentioned before, all data is returned as JSON according to a
proper schema.  This makes it easy to pull out all the information we
need.  Initially, this information was used to generate an up-to-date
listing of all MA and third-party packages to be displayed on
\url{https://modelica.org/libraries} but we also noticed quite quickly
the possibilities such an API opens up.  It was offering the very
information we needed in order to build a catalog of available
packages including the different tagged versions for download.

\subsection{GitHub only?}
\label{sec:gh-only}
The mechanism described so far seems to depend a lot on GitHub's API.
So one might wonder is there a danger of locking us to GitHub.

The answer is actually no. We chose GitHub just as \textbf{one}
possible data source.  It is possible to enhance \impact\ with other
``connectors'' to other existing package hosting solutions (private or
public).  As long as the schema is known to \impact\ it can then pull
its data from basically all possible places.  For example, it would
also be possible to use the API of the GitLab project\cite{gl-api} to
extract the same information.

\section{Package Index}
\label{sec:index}

Package information is maintained in an index file.  This index file
is also generated by \impact\ but the process of building an index is
not normally used by the user or tool vendors so all discussion about
the creation of index files is presented later in
Section~\ref{sec:private}.  The index file is stored in JSON format
and has the following structure:

{\footnotesize
\begin{verbatim}
{
  "<LibraryName>": {
    "homepage": "<URL>",
    "description": "<description string>",
    "versions": [
      "<version number>": {
        "version": "<version number>",
        "major": <major version number>,
        "minor": <minor version number>,
        "patch": <patch version number>,
        "tarball_url": "<URL to tarball>",
        "zipball_url": "<URL to zipball>",
        "path": "<path to library>",
        "dependencies": [
          {"name": "<DepLibName1>",
           "version": "<version string>"},
          {"name": "<DepLibName2>",
           "version": "<version string>"},
          ...
        ]
      }
      ...
    ]
  }
}
\end{verbatim}
}

All quantities listed within angle brackets, \code{<...>}, are library
specific details.  The \code{<LibraryName>} is the name of the
\code{package} in Modelica.  Generally speaking, all version numbers
follow the semantic versioning approach.  However, since not all
Modelica libraries currently follow semantic version conventions,
indices can include semantic duplicates (\textit{e.g.,} \code{1.0} and
\code{1.0.0}) which reference the same underlying version.  Therefore,
any non-semantic conforming versions (\textit{e.g.,} \code{1.0}) will
act as ``redirects'' to the semantic version (\textit{e.g.,}
\code{1.0.0}).

The \code{homepage} field is a URL to a web site that contains
additional information about the library.  The \code{zipball\_url} and
\code{tarball\_url} fields point to archives that can be downloaded,
in the \code{zip} and \code{tar} formats, respectively.  The
\code{dependencies} field lists all the libraries dependencies.  These
are the libraries that will also be installed when installing the
specified library version they are listed under.  The \code{path}
field specifies the name of the directory or file representing the
Modelica library within the specified archive.

Note, we have not currently defined a schema for this format.  To
promote interoperability we recognize that a formal schema would be the
next logical step.  We have added the creation of a JSON schema for
the index file format to our list of next steps to promote
interoperability with other implementations.  Our hope is that such a
schema would further encourage tool vendors to support this format as
a means of publishing information about available Modelica libraries.

The Modelica Association index of publicly available libraries can
be found at \url{https://impact.modelica.org/impact\_data.json}.

\section{Private Packages}
\label{sec:private}

\subsection{Using Private Indices}
\label{sec:use_private}

As mentioned in Section~\ref{sec:index}, there will be a package index
hosted on \href{https://modelica.org}{\code{modelica.org}} that lists
any packages connected to special Modelica related GitHub accounts.
This provides a means for library developers to quickly add their
libraries to the set of libraries that are publicly indexed.

However, we recognize that many users will depend on libraries that
cannot be hosted publicly.  At the same time, we would like for those
users to be able to benefit from the same kind of package management
features for finding and installing their privately hosted libraries.

For this reason, users can create a special configuration file that
lists the indices to be searched.  By default, \impact\ will use only
the index hosted on \href{https://modelica.org}{\code{modelica.org}}.
But through custom configurations, users can specify any collection of
indices (public or private) they wish to use.

To specify an alternative list of indices, a user would simply edit
their user configuration file and add the following text:

\begin{verbatim}
[Impact]
indices=<ur1l>,<url2>
\end{verbatim}
where the value of \code{indices} is a list of URLs pointing to index
files.  The default value for the \code{indices} variable is
\code{https://impact.modelica.org/impact\_data.json}.  In cases where
private index files used, the URLs for private index files should be
listed first and the URL to the index file hosted on modelica.org
should be last.

Note, the location of the user's configuration file will depend on the
platform their are using.  Information about the location of the
configuration file and current settings is generated by the following
command:

\begin{verbatim}
$ impact info
\end{verbatim}

\subsection{Generating Private Indices}
\label{sec:gen_private}

% MIKE

In order for users to include a private index file in the list of
indices to be searched (as discussed~\ref{sec:use_private}), it is
necessary to also have the capability to easily generate such private
indices.  This functionality is also available using the
\impact\ command line although we did not discuss it previously
because it is not functionality that a typical user would require.

To generate an index file, the following \impact\ command line syntax
should be used:

\begin{verbatim}
$ impact.py refresh <source1> \
      <source2> ... <sourcen> \
      -o <output file>
\end{verbatim}

where each source is a URL that encodes information about a potential
source.  For example, the default sources are
\code{github://modelica-3rdparty/.*} and \code{github://modelica/.*}
(in other words, all repositories belonging to the GitHub user
\code{modelica-3rdparty} and all repositories belonging to the GitHub
user \code{modelica}, respectively).  Note that later sources have a
higher priority than earlier sources.  Also, at the moment the only
types of repositories supported are GitHub repositories although by
using a URL based approach it is easy to extend the possibilities to
include Git, Subversion or other types of repositories as long as the
information required for the index file is available.

The output file generated from this command should then be made
accessible to users so they can incorporate it into the set of indices
they search (see Section~\ref{sec:use_private} for more details).

\section{Source Code and Licensing}
\label{sec:source}

The \impact\ project started off as a simple script, then a gist and
eventually a complete repository.  The repository for the source code
is hosted on GitHub at \url{https://github.com/xogeny/impact}.
Potential contributors are invited to fork the repository and add more
functionality.  Contributions to improve \impact\ are very welcome.

The software is distributed under an MIT license.  As such, there are
no significant restrictions on using the code in open-source,
closed-source or commercial projects.  In fact, we welcome vendor
support and adoption.  In addition to making the complete source code
for the package manager available and documenting the functionality
in this (freely downloadable) paper, we are also making the index data
freely available from the \href{https://modelica.org}{\code{modelica.org}}
domain.  We hope that all these measures will lead to the highest possible
chance of adoption.

\section{Conclusion}
\label{sec:conclusion}

% MIKE

Inspired by the approach taken in Bower, we've created \impact, a
package manager for the Modelica eco-system.  Like Bower, this
approach is relatively light and relies heavily on the GitHub API to
aggregate and index publicly available libraries.  Also like Bower,
our approach relies on semantic versioning, a widely adopted approach
for associating concrete meaning to the various elements of a version.
We use these meanings to help validate and organize the tags
associated with Modelica libraries.

The \impact\ tool also provides one of the key elements of a package
manager, the ability to automatically pull in dependencies during
installation.  With this feature, users can list the libraries they
directly depend on and \impact\ will automatically install any
additional dependencies.  The dependency information is constructed
automatically from the \code{version} annotation already present in
Modelica libraries.

The index of publicly available libraries is hosted on
\href{https://modelica.org}{\code{modelica.org}}.  But \impact\ can
also be used to index and install private libraries as well.  All
\impact\ functionality (installing, searching, \textit{etc.}) is
available for both public and private libraries.

\printbibliography
\end{document}
