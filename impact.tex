% sample file for Modelica Conference paper

\documentclass[11pt,a4paper,twocolumn]{article}
\usepackage{graphicx}
\graphicspath{{fig/}}
\usepackage[T1]{fontenc}
\usepackage[british]{babel}      % some british specific settings
\usepackage[utf8]{inputenc}    %% european characters can be used
\usepackage{lmodern,amsmath,mathptmx,url}      %% recommended for readable pdf
\pagestyle{empty}                %% no page numbers!
\usepackage{geometry}            %% please don't change geometry settings!
\geometry{left=20mm, right=20mm, top=25.4mm, bottom=25mm, noheadfoot, columnsep=8mm}
\parindent0pt

\usepackage[sorting=none]{biblatex}
%\bibliographystyle{ieeetr}
\bibliography{impact}

% some additional packages
\usepackage{listings} % for code listings
\usepackage{color}
\usepackage[hidelinks=true]{hyperref}

% usefull commands
\newcommand{\myr}{\textsuperscript{\textregistered}}
\newcommand{\ud}{\mathrm{d}}
\newcommand{\matx}[1]{\mathbf{#1}}
\newcommand{\impact}{\texttt{impact}} % impact is going to get used quite a lot :)
\newcommand{\code}[1]{\texttt{#1}} % make quoting code text a bit simpler

\begin{document}

\title{\textbf{{\small Modelica'2014}\\
    \impact -- A Modelica\myr\ Package Manager}}

\author{Michael Tiller\\
  \href{http://xogeny.com}{Xogeny Inc.}, USA\\
  \href{mailto:michael.tiller@xogeny.com}{\nolinkurl{michael.tiller@xogeny.com}} %
  \and Dietmar Winkler\\
  \href{http://www.hit.no}{Telemark University College}, Norway\\
  \href{mailto:dietmar.winkler@hit.no}{\nolinkurl{dietmar.winkler@hit.no}}}
\date{} % <--- leave date empty
\maketitle\thispagestyle{empty} %% <-- you need this for the first page

\section*{Abstract}

To manage complexity, modern programming languages use organizational
units to group code related by some common purpose.  Depending on the
programming language, these units might be called libraries, packages
or modules.  But they all attempt to encapsulate functionality to
promote modular code and reusability.  For the remainder of this
paper, we will simply refer to these organizational units as
\emph{packages} (as they are called in Modelica).

Also common to many modern programming languages are tools that can be
used to manage these packages.  These tools are generally called
\emph{package managers} and they allow developers to quickly ``fetch''
any packages they may need for a given project.  The main functions of
package managers are to allow developers to search, install, update
and uninstall packages with a simple command-line or graphical
interface.  In the Java world, the most common package manager is
\code{maven}.  For Python, tools like \code{easy\_install}\cite{easy_install} and
\code{pip}\cite{pip} are used for managing packages.  For client-side web
development, \code{bower} is used.  For server-side JavaScript, the
tool of choice is \code{npm}\cite{npm}.  For compiled languages, these package
managers often include some additional build functionality as well.

This paper introduces \emph{\impact}, a package manager for
Modelica. Using \impact, Modelica users and developers can quickly
search for, install and update Modelica libraries.  In this paper, we
will discuss the functionality provided by \impact.  In addition, we
will discuss how the functionality was implemented.  As part of this
we will discuss the importance of collaborative platforms, like
\href{https://github.com}{GitHub}\cite{github} in our case, for providing a means
for collecting, curating and distributing packages within a community
of developers.

The \impact\ package manager is provided to the Modelica community as
a free, open-source tool.  Furthermore, the protocols involved are all
documented and we encourage tool vendors to integrate them into their
own tools so that graphical tools can provide the same searching,
updating and installation capabilities that the command-line tool
provides.

\paragraph{Keywords:}
\emph{modelica, package manager, github, dependency management, python}

\section{Introduction}
\label{sec:intro}

It is increasingly the case that the adoption of new technologies
hinges more and more on automating away the tedious tasks required to
learn and adopt the technologies.  For programming languages or
frameworks, this means streamlining the process by which libraries can
be found and installed.

For nearly all modern languages, this issue of ``package management''
has reached the point where it is almost an element of language
design.  The Java world has the \code{maven} tool, Scala has
\code{sbt}, Node.js has \code{npm} and the Go language includes
built-in support for package management via its the command line
\code{go} compiler.

In the Modelica world, this issue has been largely overlooked.
Although there have been proposals for formats to list network
accessible libraries, these efforts have remained mere proposals
without any concrete functionality.  The \impact\ project was inspired
by the \code{Bower}\cite{Bower} project's approach.  This lightweight,
\code{git}-centric approach (discussed in Section~\ref{sec:candidates})
turned out to be relatively straightforward to implement and provides
functionality otherwise unavailable in the Modelica world.

The goal of this project is to provide the same basic package
management features found across most of the examples listed so far.
These features will be discussed in detail in subsequent sections of
this paper (see Section~\ref{sec:command_line}).  Our goal is to lower
the barrier for users to find, install and update libraries.  At the
same time, we expect that the \impact\ tool itself will be just as
easy to install as the libraries it supports.

The problem that \impact\ addresses is making installation of Modelica
packages as easy as possible.  There are actually three important
aspects to the problem.  The most obvious aspect is providing a
command-line interface that can be used by users to easily install not
just a given Modelica library, but \textbf{also its dependencies}.
However, such a command line tool must rely on the second aspect which
is the avialability of a centrally served, up to date index of
packages.  The final aspect is making it easy for library developers
\textbf{to publish} their libraries in such a way that they are
available to other Modelica users through the \impact\ package
manager.  Each of these aspects will be discussed as part of this
paper.

It is worth noting that while \impact\ can handle dependences, it
does not solve some of the problems currently inherent in Modelica.  At
the moment, dependencies between packages are described by individual
versions.  The result is that these dependencies can create brittle
chains which are not always possible to satisfy. The logic for
unconvering dependencies in \impact\ is very simple.  It merely
identifies any dependencies explicitly listed by the library and then
attempts to find that version of those packages within the
\impact\ package index. Hopefully the Modelica annotations to express
dependencies will be refined to support a richer set of relationships.
If so, the logic used by \impact\ to identify and install these
dependencies can be extended to support this improved expressiveness.

\section{Command Line Interface}
\label{sec:command_line}

\subsection{Installation of \impact}
\label{sec:install}

The \impact\ tool is available from ``PyPI''\cite{pypi} and
can be simply installed by running either

\lstset{language=bash}
% Start your code-block
\begin{lstlisting}
$ pip install impact
\end{lstlisting}
or
\begin{lstlisting}
$ easy_install impact
\end{lstlisting}

As an alternative one can also download the souces from
\url{https://github.com/xogeny/impact} or
\url{https://pypi.python.org/pypi/PyFMI},
unpack and run
\begin{lstlisting}
$ python setup.py install
\end{lstlisting}

in order to install it.

% DIETMAR

% Talk about installation (we need to get this up on PyPI, right?!)

% Talk about the various commands available with impact (search, list,
% install, etc).  Don't talk about generating indices...we'll talk about
% that later.

\subsection{Searching} % Make each command a subsection
\label{cmd:search}

Search for librararies is done by executed by doing:

\begin{lstlisting}
$ impact search <search term>
\end{lstlisting}

This will return a list of all package which names and/or description strings
contain the ``\code{<search term>}''. The returned list contains also the URL
from where the Modelica package originates.

The output can also be made more verbose with:
\begin{lstlisting}
$ impact search -v <search term>
\end{lstlisting}
which then will return the description string and the available versions
in addition.

\subsection{Installing packages} % Make each command a subsection
\label{cmd:install-pkg}
Once a package of interest is found with the use of \code{search} it can
be installed by executing:
\begin{lstlisting}
$ impact install <package name>
\end{lstlisting}
This will then fetch not only the package itself and extract it in a
configurable target directory but will also fetch the dependencies of
the packages as long as those are available to \impact.

In case several versions of one package are available, \impact will choose
the latest available. If this is not desirable then one can also specify the
version explicitely. E.g., in order to install Modelica version 2.2.2
one would do:
\begin{lstlisting}
$ impact install Modelica#2.2.2
\end{lstlisting}

Just like for the \code{search} sub-command there is also a verbose switch
``\code{-v}'' available for \code{install} this will give information on what
versions are available, which version is going to be installed, what are the
dependencies and which file is downloaded from where.

In addition there is also a ``\code{[--dry-run|-d]}'' option available which
does not actually download or extract any files but will simply report what
\impact\ \emph{would} do.
It makes sense to use the ``dry-run'' option always in combination with the
``verbose'' option.

% Discuss "refresh" command here?  I assumed no when I wrote the
% section on private repositories.  But now I'm struggling with how
% to introduce the topics in section sec:index if we haven't talked
% about how indices are made.

% I agree, we should not talk about this "plumbing" command here.

\section{Candidate Packages}
\label{sec:candidates}

% DIETMAR

% Discuss how we construct the pool of packages to consider for impact.

% Discuss semantic versioning, tags, github

% Discuss the Github API, how we walk it, what we produce from it and
% where we store the results

\section{Package Index}
\label{sec:index}

% MIKE

% This is an awkward way to introduce this topic (without already
% discussing where these index files come from)...

The index file used by \impact\ is stored in JSON format.  The index file
has the following structure:

{\footnotesize
\begin{verbatim}
{
  "<LibraryName>": {
    "homepage": "<URL>",
    "description": "<description string>",
    "versions": [
      "<version number>": {
        "version": "<version number string>",
        "major": <major version number>,
        "minor": <minor version number>,
        "patch": <patch version number>,
        "tarball_url": "<URL to tarball>",
        "zipball_url": "<URL to zipball>",
        "path": "<path within repository to library>",
        "dependencies": [
          ["<DepLibName1>", "<version string>"],
          ["<DepLibName2>", "<version string>"],
          ...
        ]
      }
      ...
    ]
  }
}
\end{verbatim}
}

All quantities listed within angle brackets, \code{<...>}, are library
specific details.  The \code{<LibraryName>} is the name of the
\code{package} in Modelica.  Generally speaking, all version numbers
follow the semantic versioning approach.  However, since not all
Modelica libraries follow semantic version conventions, indices can
include semantic duplicates (\textit{e.g.,} \code{1.0} and
\code{1.0.0}) amount the set of listed versions.  Any non-semantic
conforming versions (\textit{e.g.,} \code{1.0}) will act as
``redirects'' to the semantic version (\textit{e.g.,} \code{1.0.0}).

The \code{homepage} field is a URL to a web site that contains
additional information about the library.  The \code{zipball\_url} and
\code{tarball\_url} fields point to archives that can be downloaded,
in the \code{zip} and \code{tar} formats respectively, and extracted.
The \code{dependencies} field lists all the libraries dependencies.
These are the libraries that will also be installed when installing
the library and version they are listed under.  The \code{path} field
specifies the name of the directory or file representing the Modelica
library within the specified archive.

Note, we have not currently defined a schema for this format.  To
promote interopability we recognize that a formal schema would be the
next logical step.  We have added the creation of a JSON schema for
the index file format to our list of next steps to promote
interoperability with other implementations.  Our hope is that such a
schema would further encourage tool vendors to support this format as
a means of publishing information about avialable Modelica libraries.

The Modelica Association index of publically available libraries can
be found at \code{https://impact.modelica.org/impact\_data.json}.

\section{Private Packages}
\label{sec:private}

\subsection{Using Private Indices}
\label{sec:use_private}

As mentioned in Section~\ref{sec:index}, there will be a package index
hosted on \code{modelica.org} that lists any packages connected to
special Modelica related GitHub accounts.  This provides a means for
library developers to quickly add their libraries to the set of
libraries that are publicly indexed.

However, we recognize that many users will depend on libraries that
cannot be hosted publicly.  At the same time, we would like for those
users to be able to benefit from the same kind of package management
features for finding and installing their privately hosted libraries.

For this reason, users can create a special configuration file that
lists the indices to be searched.  By default, \impact\ will use only
the index hosted on \code{modelica.org}.  But through custom
configurations, users can specify any collection of indices (public or
private) they wish to use.

To specify an alternative list of indices, a user simply edit their
user configuration file and add the following text:

\begin{verbatim}
[Impact]
indices=<ur1l>,<url2>
\end{verbatim}
where the value of \code{indices} is a list of URLs pointing to index
files.  The default value for the \code{indices} variable is \code{}.

% Kind of lame, but I don't feel like putting the complete logic for this in
% the paper.  It might be nice to include a command-line option that just
% tells the user where it would search for their configuration file?
% (added as issue #34)

Note, the location of the user's configuration file will depend on the
platform their are using.

\subsection{Generating Private Indices}
\label{sec:gen_private}

% MIKE

In order for users to include a private index file in the list of
indices to be searched (as discussed~\ref{sec:use_private}), it is
necessary to also have the capability to easily generate such private
indices.  This functionality is also available using the
\impact\ command line although we did not discuss it previously
because it is not functionality that a typical user would require.

To generate an index file, the following \impact\ command line syntax
should be used:

\begin{verbatim}
$ impact.py refresh <source1> \
      <source2> ... <sourcen> -o <output file>
\end{verbatim}

where each source is a URL encoding information about a potential
source.  For example, the default sources are
\code{github://modelica-3rdparty/.*} and \code{github://modelica/.*}
(in other words, all repositories belonging to the GitHub user
\code{modelica-3rdparty} and all repositories belonging to the GitHub
user \code{modelica}, respectively).  Note that later sources have a
higher priority than earlier sources.  Also, at the moment the only
types of repositories supported are GitHub repositories although by
using a URL based approach it is easy to extend the possibilities to
include Git, Subversion or other types of repositories.

The output file generated from this command should then be made
accessible to users so they can incorporate it into the set of indices
they search (see Section~\ref{sec:use_private} for more details).

\section{Source Code and Licensing}
\label{sec:source}

The \impact\ project started off a simple script, then a gist and
eventually a complete repository.  The repository for the source code
is hosted on GitHub at \url{https://github.com/xogeny/impact}.
Potential contributors are invited to fork the repository and add more
functionality.

The software is distributed under an MIT license.  As such, there are
no significant restrictions on using the code in open-source,
closed-source or commerical projects.  In fact, we welcome vendor
support and adoption.  In addition to making the complete source code
for the package manager available and documentating the functionality
in this (freely downloadable) paper, we are also making the index data
freely available from the modelica.org domain.  We hope that all these
measures will lead to the highest possible chance of adoption.

\section{Conclusion}
\label{sec:conclusion}

% MIKE

% Relatively light implementation modeled on the Bower approach

% Leverages Git (and/or potentially other version control systems, if
% people want to add support) to build package index

% Relies on semantic versioning

% Handles dependencies
\printbibliography
\end{document}
